\documentclass{article}
\usepackage[utf8]{inputenc}
\usepackage{geometry}
\usepackage{graphicx} % Include this line
\geometry{a4paper, margin=1in}

\title{
  \textbf{LAPORAN PRAKTIKUM}\\
  \textbf{PEMROGRAMAN BERBASIS OBJEK}\\
  \textbf{INHERITANCE}
}
\author{} % Jika Anda ingin memasukkan nama penulis, tulis di sini.
\date{} % Jika Anda ingin memasukkan tanggal, tulis di sini.

\begin{document}
\maketitle

\begin{figure}[h]
    \centering
    \includegraphics[width=0.8\textwidth]{Logo_Universitas_Pertamina.png}
\end{figure}

\begin{center}
\fontsize{14pt}{16pt}\selectfont Disusun oleh:\\
\fontsize{14pt}{16pt}\selectfont Rayhan Surya Destian (105222024)\\
\Large \textbf{ }\\
\Large \textbf{PROGRAM STUDI ILMU KOMPUTER}\\
\Large \textbf{FAKULTAS SAINS DAN KOMPUTER}\\
\Large \textbf{PEMROGRAMAN BERORIENTASI OBJEK}\\
\Large \textbf{UNIVERSITAS PERTAMINA}\\
\Large \textbf{2023/2024}
\end{center}

\newpage

\section*{I. Pendahuluan}
Program ini adalah simulasi game petualangan dungeon di mana pemain dapat berinteraksi melalui menu untuk melakukan login, register, menjelajahi dungeon, player bisa bertarung dengan musuh dan juga mendapatkan item yang bisa digunakan.

\section*{II. Variabel}

Tabel variabel untuk kelas \texttt{Game}, \texttt{Player}, \texttt{Item}, dan \texttt{Enemy} dalam program:

\begin{tabular}{|c|l|l|p{7cm}|}
\hline
No & Nama Variabel & Tipe Data & Fungsi \\
\hline
1 & accountManager & AccountManager & Mengelola login dan register player \\
2 & player & Player & Menyimpan data dan status pemain dalam game \\
3 & isRunning & boolean & Mengontrol loop utama game \\
4 & id & String & Identifikasi unik untuk item dan musuh \\
5 & name & String & Nama dari item atau musuh \\
6 & description & String & Deskripsi item \\
7 & type & String & Tipe item \\
8 & stats & String & Statistik yang ditawarkan item \\
9 & value & int & Nilai efek dari item \\
10 & health & int & Darah musuh atau pemain \\
11 & maxHealth & int & Darah maksimum musuh atau pemain \\
12 & damage & int & Damage yang diberikan oleh musuh \\
13 & experience & int & Experience yang diperoleh dari mengalahkan musuh \\
\hline
\end{tabular}

\section*{III. Constructor dan Method}

Tabel constructor dan method yang signifikan dalam program:

\begin{tabular}{|c|l|l|p{7cm}|}
\hline
No & Nama Method & Jenis Method & Fungsi \\
\hline
1 & Game() & Constructor & Membuat objek Game dengan inisialisasi variabel \texttt{accountManager}, \texttt{player}, dan \texttt{isRunning} \\
2 & run() & Procedural & Menjalankan loop utama game untuk interaksi menu utama \\
3 & login(Scanner) & Functional & Memproses login pengguna \\
4 & register(Scanner) & Procedural & Memproses registrasi pengguna \\
5 & gameMenu(Scanner) & Procedural & Menampilkan dan mengelola menu permainan setelah login \\
6 & venture(Scanner) & Procedural & Mengelola eksplorasi dungeon oleh pemain \\
7 & encounter(Scanner) & Procedural & Mengelola pertemuan secara acak dengan musuh atau temuan item \\
8 & combat(Scanner, Enemy) & Procedural & Melaksanakan pertarungan antara pemain dan musuh \\
9 & checkInventory() & Procedural & Menampilkan dan mengelola inventaris pemain \\
10 & Item(...) & Constructor & Membuat objek Item dengan atribut seperti nama, deskripsi, dan lainnya \\
11 & use(Player) & Functional & Menerapkan efek item pada pemain \\
12 & Enemy(...) & Constructor & Membuat objek Enemy dengan atribut seperti health, damage, dan pengalaman \\
13 & takeDamage(int) & Functional & Mengurangi health musuh berdasarkan damage yang diterima \\
14 & isAlive() & Functional & Memeriksa apakah musuh masih hidup atau tidak \\
\hline
\end{tabular}

\section*{IV. Dokumentasi dan Pembahasan Code}

Program game dungeon ini dirancang menggunakan paradigma pemrograman berorientasi objek (OOP), yang memfasilitasi pembagian dan manajemen tanggung jawab melalim pemisahan fungsi-fungsi dalam kelas-kelas yang terdefinisi dengan jelas. 

\paragraph{Inisialisasi dan Manajemen Pengguna:}
Dalam kelas \texttt{Game}, constructor digunakan untuk menginisialisasi objek-objek penting seperti \texttt{accountManager} dan \texttt{player}. Variabel \texttt{isRunning} diatur untuk mengontrol loop utama permainan, sementara method \texttt{login()} dan \texttt{register()} menangani interaksi pengguna untuk proses autentikasi.

\paragraph{Navigasi dan Encounter Dungeon:}
Fungsi \texttt{venture(Scanner)} memberikan simulasi adventure dungeon oleh pemain, di mana tindakan pemain dapat memicu pertemuan acak yang diatur oleh method \texttt{encounter(Scanner)}. Mekanisme ini menunjukkan penggunaan efektif dari polymorphism dan inheritance melalui kelas abstrak \texttt{Item} dan \texttt{Enemy}, yang memungkinkan perluasan fungsionalitas secara dinamis tanpa perlu mengubah kode yang telah ada.

\paragraph{Mekanisme Pertarungan:}
Dalam pertarungan, logika interaksi pengguna seperti serangan, penggunaan item, dan pelarian diatur melalui \texttt{combat(Scanner, Enemy)}. Metode ini menunjukkan bagaimana manajemen state dinamis dapat diimplementasikan dalam OOP untuk responsif terhadap aksi pengguna. Implementasi ini juga menunjukkan bagaimana aplikasi dapat diuji dan diperluas dengan mudah berkat modularitas yang dihasilkan oleh pendekatan berorientasi objek.


\section*{V. Kesimpulan}

Program game adventure dungeon ini menggabungkan prinsip-prinsip OOP untuk membangun aplikasi yang modular dan bisa dibilang cukup mudah dikelola. Penggunaan kelas abstrak, inheritance, dan polymorphism memberikan kerangka kerja yang lumayan solid untuk pengembangan lebih lanjut. Walaupun memiliki banyak area yang mungkin dapat Saya perbaiki dan tambah, struktur yang sudah Saya bangun memberikan dasar yang kuat untuk iterasi selanjutnya dan peningkatan fungsionalitas.


\end{document}
